\documentclass[14pt,eqno, fontsize=14pt]{article}

\usepackage{physics}
\usepackage{latexsym} 
\usepackage[left=30mm, top=20mm, right=15mm, bottom=20mm, nohead, nofoot]{geometry}


%%% Дополнительная работа с математикой
\usepackage{amsmath,amsfonts,amssymb,amsthm,mathtools} % AMS
\usepackage{icomma} % "Умная" запятая: $0,2$ --- число, $0, 2$ 

%%% Работа с русским языком
\usepackage{cmap}					% поиск в PDF
%\usepackage{mathtext} 				% русские буквы в формулах
\usepackage[T2A]{fontenc}			% кодировка
\usepackage[utf8]{inputenc}			% кодировка исходного текста
\usepackage[english,russian]{babel}	% локализация и переносы

%%% Работа с картинками
\usepackage{graphicx}  % Для вставки рисунков
\graphicspath{{image/}}  % папки с картинками
\setlength\fboxsep{3pt} % Отступ рамки \fbox{} от рисунка
\setlength\fboxrule{1pt} % Толщина линий рамки \fbox{}
\usepackage{wrapfig} % Обтекание рисунков и таблиц текстом


\usepackage[backend=biber,style=numeric,sorting=none]{biblatex}
\usepackage{csquotes}
\addbibresource{bibleograf.bib}


\linespread{1.3}
%\pagestyle{plain}

\begin{document}
\begin{center}
\small
Федеральное государственное автономное образовательное учреждение высшего образования\\
<<Московский физико-технический институт (Национальный исследовательский университет)>> \\
Физтех-школа аэрокосмических технологий\\
Кафедра теоретической и экспериментальной физики геосистем
\end{center}
~\
\begin{flushleft}
\small
\textbf{Направление подготовки:} 03.03.01 Прикладные математика и физика (бакалавриат)\\
\textbf{Направленность(профиль) подготовки:} Физика и механика космических и природных систем\\
\end{flushleft}
~\
~\

~\
~\
~\
~\
~\

~\
~\
~\

~\

~\
~\
\begin{center}
\LARGE
\textbf{Пространственно-временное распределение полного электронного содержания в различных геофизических условиях}\\
~\
\small(бакалаврская работа)
\end{center}

~\

~\

~\

~\

\begin{flushleft}

\hspace{300pt} \textbf{Студент:}\\
\hspace{300pt} Скачков Алексей Павлович\\
\hspace{300pt} \underline{\hspace{3cm}}\\
\hspace{300pt} \textbf{Научный руководитель:}\\
\hspace{300pt} Ряховский Илья Александрович\\
\hspace{300pt} \underline{\hspace{3cm}}\\

\end{flushleft}

~\

~\

~\

~\

\begin{center}
\small
Москва\\
2020
\end{center}

%%%%%%%%%%%%%%%%%%%%%%%
\newpage
\section*{Аннотация}
\subsection*{Цели и задачи}

\subsection*{Полученные результаты}

\newpage
\tableofcontents

\newpage
\section*{Используемые обозначения}

\newpage
\section*{Введение}
\subsection*{Актуальность темы}
Исследование ионосферы является достаточно важным направлением, так как от ее состояния зависит множество факторов, влияющих на нашу повседневную жизнь. Знание о состоянии ионосферы может помогать идентифицировать различные события техногенного и естественного характеров. В современной действительности стало ясно, что различные ионосферные процессы влияют на погодные и климатические условия. Не стоит забывать и о современных средствах связи, навигации и локации, которые напрямую зависят от состояния ионосферы.

\subsection*{Объект исследования}
Основные параметры, характеризующие ионосферу: локальная электронная концентрация $N_e$, температура ионов и электронов и полное электронное содержание.

Объектом исследования данной работы является полное электронное содержание (ПЭС или TEC в англоязычной литературе). ПЭС представляет собой количество электронов в столбе единичного сечения.

\subsection*{Значимость исследования}


\newpage

\end{document}