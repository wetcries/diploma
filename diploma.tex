\documentclass[14pt,eqno, fontsize=14pt]{article}

\usepackage{physics}
\usepackage{latexsym} 
\usepackage[left=30mm, top=20mm, right=15mm, bottom=20mm, nohead, nofoot]{geometry}


%%% Дополнительная работа с математикой
\usepackage{amsmath,amsfonts,amssymb,amsthm,mathtools} % AMS
\usepackage{icomma} % "Умная" запятая: $0,2$ --- число, $0, 2$ 

%%% Работа с русским языком
\usepackage{cmap}					% поиск в PDF
%\usepackage{mathtext} 				% русские буквы в формулах
\usepackage[T2A]{fontenc}			% кодировка
\usepackage[utf8]{inputenc}			% кодировка исходного текста
\usepackage[english,russian]{babel}	% локализация и переносы

%%% Добавлена поддержка для кастомизации таблиц
\usepackage[table,xcdraw]{xcolor}
\usepackage{booktabs}

%%% Работа с картинками
\usepackage{graphicx}  % Для вставки рисунков
\graphicspath{{image/}}  % папки с картинками
\setlength\fboxsep{3pt} % Отступ рамки \fbox{} от рисунка
\setlength\fboxrule{1pt} % Толщина линий рамки \fbox{}
\usepackage{wrapfig} % Обтекание рисунков и таблиц текстом


\usepackage[backend=biber,style=numeric,sorting=none]{biblatex}
\usepackage{csquotes}
\addbibresource{bibleograf.bib}


\linespread{1.3}
%\pagestyle{plain}

\begin{document}
\begin{center}
\small
Федеральное государственное автономное образовательное учреждение высшего образования\\
<<Московский физико-технический институт (Национальный исследовательский университет)>> \\
Физтех-школа аэрокосмических технологий\\
Кафедра теоретической и экспериментальной физики геосистем
\end{center}
~\
\begin{flushleft}
\small
\textbf{Направление подготовки:} 03.03.01 Прикладные математика и физика (бакалавриат)\\
\textbf{Направленность(профиль) подготовки:} Физика и механика космических и природных систем\\
\end{flushleft}
~\
~\

~\
~\
~\
~\
~\

~\
~\
~\

~\

~\
~\
\begin{center}
\LARGE
\textbf{Пространственно-временное распределение полного электронного содержания в различных геофизических условиях}\\
~\
\small(бакалаврская работа)
\end{center}

~\

~\

~\

~\

\begin{flushleft}

\hspace{300pt} \textbf{Студент:}\\
\hspace{300pt} Скачков Алексей Павлович\\
\hspace{300pt} \underline{\hspace{3cm}}\\
\hspace{300pt} \textbf{Научный руководитель:}\\
\hspace{300pt} Ряховский Илья Александрович\\
\hspace{300pt} \underline{\hspace{3cm}}\\

\end{flushleft}

~\

~\

~\

~\

\begin{center}
\small
Москва\\
2020
\end{center}

%%%%%%%%%%%%%%%%%%%%%%%
\newpage
\section*{Аннотация}
\subsection*{Цели и задачи}

\subsection*{Полученные результаты}

\newpage
\tableofcontents

\newpage
\section*{Используемые обозначения}

\newpage
\section*{Введение}
\addcontentsline{toc}{section}{Введение}
\subsection*{Актуальность темы}
Исследование ионосферы является достаточно важным направлением, так как от ее состояния зависит множество факторов, влияющих на нашу повседневную жизнь. Знание о состоянии ионосферы может помогать идентифицировать различные события техногенного и естественного характеров. В современной действительности стало ясно, что различные ионосферные процессы влияют на погодные и климатические условия. Не стоит забывать и о современных средствах связи, навигации и локации, которые напрямую зависят от состояния ионосферы.

\subsection*{Объект исследования}
Основные параметры, характеризующие ионосферу: локальная электронная концентрация $N_e$, температура ионов и электронов и полное электронное содержание.

Объектом исследования данной работы является полное электронное содержание (ПЭС или TEC в англоязычной литературе). ПЭС представляет собой количество электронов в столбе единичного сечения. В рамках данной работы предлагается получение пространственно-временного распределения полного электронного содержания во время высокой солнечной активности.

\subsection*{Значимость исследования}


\newpage
\section{Теоретические сведения}
\subsection{Использование GPS в исследовании ионосферы}
Существует множество различных методов, применяемых для исследования состояния ионосферы, такие как вертикальное, наклонное, вертикально-наклонное, внешнее зондирования, некогерентное рассеяние и многие другие. Появление глобальной навигационной системы и создание огромной сети GPS станций стали началом новой эры дистанционного исследования ионосферы. Большое количество станций и непрерывная доступность спутников позволяют производить своевременный мониторинг ионосферы в различных участках планеты. 

\subsection{Общие сведения о GPS}
GPS (Global Positioning System) представляет из себя спутниковую систему навигации, которая обеспечивает измерение расстояния между спутником и приемником, а так же времени. На основе этих данных определяется местоположение объекта в пространстве.

Систему GPS можно разделить на три основные составляющие:
\begin{itemize}
\item Космический сегмент
\item Сегмент управления
\item Сегмент потребителей
\end{itemize}

\textbf{Космический сегмент} состоит из $32$ спутников (один из которых находится на этапе развертки)\footnote{на момент Февраля 2019 года}, которые размещены на шести круговых орбитах. Высота орбит составляет $20200$ км. Наклонение орбит также являет общим и равно $55^{\circ}$. Каждая орбита разнесена друг относительно друга на $60^{\circ}$ по долготе. Спутники оборудованы специальным устройством, которое хранит системное время аппарата. Временные шкалы всех спутников согласованы между собой и синхронизируются системой единого времени.

Спутники непрерывно передают сигналы на двух частотах: $f_1 = 1575.42 \text{ МГц}$ и $f_2 = 1227.60 \text{ МГц}$. Передаваемые сигналы модулируются псевдослучайными последовательностями (PRN - Pseudorandom Noise) двух типов C/A-код и P-код.

C/A-код является открытым кодом, который, в основном, используется в гражданских целях. Он имеет длину повторения 1 мс и частоту следования импульсов $1.023 \text{ МГц}$.

P-код - это защищенный код. Частота следования имеет значение $10.23 \text{ МГц}$ и длину в 267 суток. Сигналы, модулированные P-кодом, передаются на двух частотах $f_1$ и $f_2$, в то время как C/A-код только на $f_1$.

Вместе с PRN-кодами также отправляются навигационные сообщения, которые содержат данные о положении спутника, метки времени, частотно-временные поправки, сведения о работоспособности оборудования и др.

\textbf{Сегмент управления} осуществляет слежение за орбитальными аппаратами и управление ими. Главная станция находится в Колорадо-Спрингс, штат Колорадо. Станции слежения выполняют измерения траекторий по сигналам спутников и после корректируют поведение каждого спутника.

\textbf{Сегмент потребителей} состоит из устройств разной степени сложности, от военного оборудования до гражданских мобильных устройств. GPS-приемники производят выбор рабочего созвездия (набора из не менее 4 видимых спутников), поиск, слежение и декодировку входящего сигнала, обработку измеряемых радионавигационных параметров и служебной информации, расчет координат и скорости потребителя.

\subsection{Интересующие виды измерений в GPS}
Основная величина, которая измеряется в спутниковых системах позиционирования, является <<псевдодальность>>, через которую определяют координаты GPS-приемника.

\begin{equation}
D' = \sqrt{(x - x_S)^2 + (y - y_S)^2 + (z - z_S)^2} + c\tau_R +\sigma_D,
\end{equation}
где $D'$ - <<псевдодальнось>> между приемником и спутником; $x_S, y_S, z_S$ -- координаты спутника; $x, y, z$ -- координаты приемника; $c$ -- скорость света; $\tau_R$ - отклонение часов приемника от системного времени GPS; $\sigma_D$ -- погрешность измерения. 
Псевдодальность отличается от действительного расстояния $D = \sqrt{(x - x_S) ^ 2 + (y - y_S) ^ 2 + (z - z_S) ^ 2}$ наличием ошибок измерений. 
Зная значения псевдодальности для 4 спутников, можно вычислить координаты приемника и значение $\tau_R$. Нахождение данных величин возможно в любой момент времени, так как в поле зрения приемника всегда оказывается минимум 5 спутников. В современных устройствах для вычисления положения в пространстве используется метод взвешенных наименьших квадратов. Для определения псевдодальности измеряются такие параметры, как время распространения сигнала и набег фазы несущей радиоволны на трассе <<спутник -- приемник>>. В зависимости от выбранного параметра различают кодовые и фазовые измерения псевдодальности.

\textbf{Кодовые измерения псевдодальности.} $D' = c \tau$. В данном случае измеряется время задержки между моментом излучения и момента получения сигнала, т.е. время распространения сигнала. Для измерения задержки, с помощью корреляционного анализа, определяется сдвиг выбранного кода, посланного спутником, относительно кода, генерируемого приемным устройством. Таким образом, двухчастотный приемник имеет возможность измерять псевдодальность тремя способами: с помощью C/A-кода на частоте $f_1$ и по P-коду на частотах $f_1$ и $f_2$\footnote{измерение по C/A-коду обозначается как $C1$, а для P-кода соответственно $P1$ и $P2$}. Точность определения псевдодальности по кодовым измерениям составляет $1\%$ от длины кода, что позволяет делать измерение по C/A-коду с погрешностью в 3 метра, а по P-коду c погрешностью 0.3 метра.

\textbf{Фазовые измерения псевдодальности.} $D' = \lambda \Delta \varphi + \lambda \text{N}$. Для получения пседодальности в этом случае измеряется разность фаз $\Delta\varphi$ двух несущих радиоволн: принятой приемником и сгенерированной в самом приемнике; $\lambda = c / f$ -- длина волны несущей частоты. Для фазовых измерений на частотах $f_1$ и $f_2$ приняты обозначения $L1$ и $L2$ соответственно. Полное число циклов фазы N остается неизвестной величиной. Этому дали название <<фазовой неоднозначностью измерений>>. Для ее устранения существует ряд способов, одним из которых является комбинирование кодовых и фазовых измерений. Погрешность измеренной разности фаз $\Delta\varphi$ имеет точность до 0.01 периода. Тогда псевдодальность может быть определена с точностью до 1-2 мм.

\textbf{Погрешности измерений.} На точность измерений влияет множество факторов, которые представлены в таблице.

% Таблица с данными о погрешности
\begin{table}[]
\begin{tabular}{|l|l|}
\hline
\rowcolor[HTML]{DAE8FC} 
{\color[HTML]{000000} \textbf{Источник погрешности}}                                                                                     & {\color[HTML]{000000} \textbf{Вносимая погрешность}} \\ \hline
Геометрическое расположение НИСЗ                                                                                                         & PDOP                                                 \\ \hline
Неточности расчетов орбит НИСЗ и времени                                                                                                 & 0.5 -- 3 м                                           \\ \hline
Случайные отклонения опбит и часов                                                                                                       & 0.5 -- 3 м                                           \\ \hline
Шумы приемника                                                                                                                           & 1.5 -- 3 м                                           \\ \hline
Задержка сигнала в ионосфере                                                                                                             & 2 -- 10 м                                            \\ \hline
Задержка сигнала в тропосфере                                                                                                            & 1 -- 2 м                                             \\ \hline
\begin{tabular}[c]{@{}l@{}}Многолучевость распространения\\ (в результате отражений от крупных объектов\\ вблизи приемника)\end{tabular} & 1 -- 2 м                                             \\ \hline
\begin{tabular}[c]{@{}l@{}}Меры по искусственному снижению точности \\ (с Мая 2000 года не используется)\end{tabular}                    & до 30 м                                              \\ \hline
Прочие источники                                                                                                                         & 1 м                                                  \\ \hline
\end{tabular}
\end{table}

Наиболее важным фактором для получения хорошей точности является геометрия рабочего созвездия спутников. Для характеристики взаимного расположения приемника и спутника вводится коэффициент PDOP (Position Dilution of Precision)\footnote{Величина PDOP обратно пропорциональна объему фигуры, образованной пересечение лучей <<спутник -- приемник>> со сферой единичного радиуса, центр которой совмещен с приемником.}. На данный коэффициент умножается все другие ошибки. 

Вторым по значимости фактором, снижающим точность, является ионосферная задержка радиосигнала. Именно из-за этого эффекта GPS может использоваться для исследования состояния ионосферы.

Для снижения ионосферной и тропосферной погрешностей измерений используются математические модели, комбинирование данных, сглаживание данных и режим DGPS\footnote{суть метода заключается в том, что измерения производятся двумя приемниками, один из которых неподвижен (для него известно истинное положение). Неподвижный приемник сравнивает свое истинное положение с положением, полученным с GPS, и отправляет поправочные коэффициенты второму приемнику.}.

Комбинация кодовых и фазовых измерений и использование их в алгоритмах сглаживания данных позволяют эффективно фильтровать погрешности, связанные с геометрией рабочего созвездия, шумами приемника, случайными отклонениями орбит часов и многолучевостью. 

\subsection{Геометрические положения, используемые для GPS зондирования}

 

 

\end{document}
