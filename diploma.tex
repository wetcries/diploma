\documentclass[14pt]{article}

\usepackage{extsizes}
\usepackage{physics}
\usepackage{latexsym} 
\usepackage[left=30mm, top=20mm, right=15mm, bottom=20mm]{geometry}
\usepackage{indentfirst}

%%% Дополнительная работа с математикой
\usepackage{amsmath,amsfonts,amssymb,amsthm,mathtools} % AMS
\usepackage{icomma} % "Умная" запятая: $0,2$ --- число, $0, 2$ 

%%%% Работа с русским языком
\usepackage{cmap}					% поиск в PDF
\usepackage{mathtext} 				% русские буквы в формулах
\usepackage[T2A]{fontenc}			% кодировка
\usepackage[utf8]{inputenc}			% кодировка исходного текста
\usepackage[english,russian]{babel}	% локализация и переносы


%%% Добавлена поддержка для кастомизации таблиц
\usepackage[table,xcdraw]{xcolor}
\usepackage{booktabs}

%%% Работа с картинками
\usepackage{graphicx}  % Для вставки рисунков
\graphicspath{{image/}}  % папки с картинками
\setlength\fboxsep{3pt} % Отступ рамки \fbox{} от рисунка
\setlength\fboxrule{1pt} % Толщина линий рамки \fbox{}
\usepackage{wrapfig} % Обтекание рисунков и таблиц текстом


\usepackage[backend=biber,style=numeric,sorting=none]{biblatex}
\usepackage{csquotes}
%\addbibresource{bibleograf.bib}


\linespread{1.3}
%\pagestyle{plain}

% biber
\usepackage[backend=biber,style=numeric,sorting=none]{biblatex}
\usepackage{csquotes}
\addbibresource{books.bib}

%%%%%%%%%%%%%%%%%%%%%%%%%%%%%%%%%%%%%%%%%%%%%%%%%%%%%%%%%%%%%%%%%%%%%%%%%%%%%%%%%%%%%%%%%%%

\begin{document}
\begin{center}
\small
Федеральное государственное автономное образовательное учреждение высшего образования\\
<<Московский физико-технический институт (Национальный исследовательский университет)>> \\
Физтех-школа аэрокосмических технологий\\
Кафедра теоретической и экспериментальной физики геосистем
\end{center}

\begin{flushleft}
\small
\textbf{Направление подготовки:} 03.03.01 Прикладные математика и физика (бакалавриат)\\
\textbf{Направленность(профиль) подготовки:} Физика и механика космических и природных систем\\
\end{flushleft}

\begin{center}
\LARGE
\textbf{Пространственно-временное распределение полного электронного содержания в различных геофизических условиях}

\small(бакалаврская работа)
\end{center}


\begin{flushright}

\noindent
\textbf{Студент:} \\
Скачков Алексей Павлович\\
\underline{\hspace{3cm}}\\
\textbf{Научный руководитель:}\\
Ряховский Илья Александрович\\
\underline{\hspace{3cm}}\\

\end{flushright}

\begin{center}
\small
Москва\\
2020
\end{center}

%%%%%%%%%%%%%%%%%%%%%%%%%%%%%%%%%%%%%%%%%%%

\newpage
\section*{Аннотация}
\subsection*{Цели и задачи}

\subsection*{Полученные результаты}

\newpage
\tableofcontents

\newpage
\section*{Используемые обозначения}

\newpage
\section*{Введение}
\addcontentsline{toc}{section}{Введение}
\subsection*{Актуальность темы}
Исследование ионосферы является достаточно важным направлением, так как от ее состояния зависит множество факторов, влияющих на нашу повседневную жизнь. Знание о состоянии ионосферы может помогать идентифицировать различные события техногенного и естественного характеров. В современной действительности стало ясно, что различные ионосферные процессы влияют на погодные и климатические условия. Не стоит забывать и о современных средствах связи, навигации и локации, которые напрямую зависят от состояния ионосферы.

\subsection*{Объект исследования}
Основные параметры, характеризующие ионосферу: локальная электронная концентрация $N_e$, температура ионов и электронов и полное электронное содержание.

Объектом исследования данной работы является полное электронное содержание (ПЭС или TEC в англоязычной литературе). ПЭС представляет собой количество электронов в столбе единичного сечения. В рамках данной работы предлагается получение пространственно-временного распределения полного электронного содержания во время высокой солнечной активности.

\subsection*{Значимость исследования}


\newpage
\section{Теоретические сведения}
\subsection{Использование GPS в исследовании ионосферы}
Существует множество различных методов, применяемых для исследования состояния ионосферы, такие как вертикальное, наклонное, вертикально-наклонное, внешнее зондирования, некогерентное рассеяние и многие другие. Появление глобальной навигационной системы и создание огромной сети GPS станций стали началом новой эры дистанционного исследования ионосферы. Большое количество станций и непрерывная доступность спутников позволяют производить своевременный мониторинг ионосферы в различных участках планеты. 

\subsection{Общие сведения о GPS}
GPS (Global Positioning System) представляет из себя спутниковую систему навигации, которая обеспечивает измерение расстояния между спутником и приемником, а так же времени. На основе этих данных определяется местоположение объекта в пространстве.

Систему GPS можно разделить на три основные составляющие:
\begin{itemize}
\item Космический сегмент
\item Сегмент управления
\item Сегмент потребителей
\end{itemize}

\textbf{Космический сегмент} состоит из $32$ спутников (один из которых находится на этапе развертки)\footnote{на момент Февраля 2019 года \cite{gpsgov}}, которые размещены на шести круговых орбитах. Высота орбит составляет $20200$ км. Наклонение орбит также являет общим и равно $55^{\circ}$. Каждая орбита разнесена друг относительно друга на $60^{\circ}$ по долготе. Спутники оборудованы специальным устройством, которое хранит системное время аппарата. Временные шкалы всех спутников согласованы между собой и синхронизируются системой единого времени.

Спутники непрерывно передают сигналы на двух частотах: $f_1 = 1575.42 \text{ МГц}$ и $f_2 = 1227.60 \text{ МГц}$. Передаваемые сигналы модулируются псевдослучайными последовательностями (PRN - Pseudorandom Noise) двух типов C/A-код и P-код.

C/A-код является открытым кодом, который, в основном, используется в гражданских целях. Он имеет длину повторения 1 мс и частоту следования импульсов $1.023 \text{ МГц}$.

P-код - это защищенный код. Частота следования имеет значение $10.23 \text{ МГц}$ и длину в 267 суток. Сигналы, модулированные P-кодом, передаются на двух частотах $f_1$ и $f_2$, в то время как C/A-код только на $f_1$.

Вместе с PRN-кодами также отправляются навигационные сообщения, которые содержат данные о положении спутника, метки времени, частотно-временные поправки, сведения о работоспособности оборудования и др.

\textbf{Сегмент управления} осуществляет слежение за орбитальными аппаратами и управление ими. Главная станция находится в Колорадо-Спрингс, штат Колорадо. Станции слежения выполняют измерения траекторий по сигналам спутников и после корректируют поведение каждого спутника.

\textbf{Сегмент потребителей} состоит из устройств разной степени сложности, от военного оборудования до гражданских мобильных устройств. GPS-приемники производят выбор рабочего созвездия (набора из не менее 4 видимых спутников), поиск, слежение и декодировку входящего сигнала, обработку измеряемых радионавигационных параметров и служебной информации, расчет координат и скорости потребителя.

\subsection{Интересующие виды измерений в GPS}
Основная величина, которая измеряется в спутниковых системах позиционирования, является <<псевдодальность>>, через которую определяют координаты GPS-приемника.

\begin{equation}
D' = \sqrt{(x - x_S)^2 + (y - y_S)^2 + (z - z_S)^2} + c\tau_R +\sigma_D,
\end{equation}
где $D'$ - <<псевдодальнось>> между приемником и спутником; $x_S, y_S, z_S$ -- координаты спутника; $x, y, z$ -- координаты приемника; $c$ -- скорость света; $\tau_R$ - отклонение часов приемника от системного времени GPS; $\sigma_D$ -- погрешность измерения. 
Псевдодальность отличается от действительного расстояния $D = \sqrt{(x - x_S) ^ 2 + (y - y_S) ^ 2 + (z - z_S) ^ 2}$ наличием ошибок измерений. 
Зная значения псевдодальности для 4 спутников, можно вычислить координаты приемника и значение $\tau_R$. Нахождение данных величин возможно в любой момент времени, так как в поле зрения приемника всегда оказывается минимум 5 спутников. В современных устройствах для вычисления положения в пространстве используется метод взвешенных наименьших квадратов. Для определения псевдодальности измеряются такие параметры, как время распространения сигнала и набег фазы несущей радиоволны на трассе <<спутник -- приемник>>. В зависимости от выбранного параметра различают кодовые и фазовые измерения псевдодальности.

\textbf{Кодовые измерения псевдодальности.} $D' = c \tau$. В данном случае измеряется время задержки между моментом излучения и момента получения сигнала, т.е. время распространения сигнала. Для измерения задержки, с помощью корреляционного анализа, определяется сдвиг выбранного кода, посланного спутником, относительно кода, генерируемого приемным устройством. Таким образом, двухчастотный приемник имеет возможность измерять псевдодальность тремя способами: с помощью C/A-кода на частоте $f_1$ и по P-коду на частотах $f_1$ и $f_2$\footnote{измерение по C/A-коду обозначается как $C1$, а для P-кода соответственно $P1$ и $P2$}. Точность определения псевдодальности по кодовым измерениям составляет $1\%$ от длины кода, что позволяет делать измерение по C/A-коду с погрешностью в 3 метра, а по P-коду c погрешностью 0.3 метра.

\textbf{Фазовые измерения псевдодальности.} $D' = \lambda \Delta \varphi + \lambda \text{N}$. Для получения пседодальности в этом случае измеряется разность фаз $\Delta\varphi$ двух несущих радиоволн: принятой приемником и сгенерированной в самом приемнике; $\lambda = c / f$ -- длина волны несущей частоты. Для фазовых измерений на частотах $f_1$ и $f_2$ приняты обозначения $L1$ и $L2$ соответственно. Полное число циклов фазы N остается неизвестной величиной. Этому дали название <<фазовой неоднозначностью измерений>>. Для ее устранения существует ряд способов, одним из которых является комбинирование кодовых и фазовых измерений. Погрешность измеренной разности фаз $\Delta\varphi$ имеет точность до 0.01 периода. Тогда псевдодальность может быть определена с точностью до 1-2 мм.

\textbf{Погрешности измерений.} На точность измерений влияет множество факторов, которые представлены в таблице \ref{tableErrors} \cite{hoffmanErrors}, \cite{shebshaevich}.

% Таблица с данными о погрешности
\begin{table}[h]
\begin{center}
\begin{tabular}{|l|l|}
\hline
\rowcolor[HTML]{DAE8FC} 
{\color[HTML]{000000} \textbf{Источник погрешности}}                                                                                     & {\color[HTML]{000000} \textbf{Вносимая погрешность}} \\ \hline
Геометрическое расположение НИСЗ                                                                                                         & PDOP                                                 \\ \hline
Неточности расчетов орбит НИСЗ и времени                                                                                                 & 0.5 -- 3 м                                           \\ \hline
Случайные отклонения опбит и часов                                                                                                       & 0.5 -- 3 м                                           \\ \hline
Шумы приемника                                                                                                                           & 1.5 -- 3 м                                           \\ \hline
Задержка сигнала в ионосфере                                                                                                             & 2 -- 10 м                                            \\ \hline
Задержка сигнала в тропосфере                                                                                                            & 1 -- 2 м                                             \\ \hline
\begin{tabular}[c]{@{}l@{}}Многолучевость распространения\\ (в результате отражений от крупных объектов\\ вблизи приемника)\end{tabular} & 1 -- 2 м                                             \\ \hline
\begin{tabular}[c]{@{}l@{}}Меры по искусственному снижению точности \\ (с Мая 2000 года не используется)\end{tabular}                    & до 30 м                                              \\ \hline
Прочие источники                                                                                                                         & 1 м                                                  \\ \hline
\end{tabular}
\end{center}
\caption{Составляющие погрешности навигационных определений}
\label{tableErrors}
\end{table}

Наиболее важным фактором для получения хорошей точности является геометрия рабочего созвездия спутников. Для характеристики взаимного расположения приемника и спутника вводится коэффициент PDOP (Position Dilution of Precision)\footnote{Величина PDOP обратно пропорциональна объему фигуры, образованной пересечение лучей <<спутник -- приемник>> со сферой единичного радиуса, центр которой совмещен с приемником.}. На данный коэффициент умножается все другие ошибки. 

Вторым по значимости фактором, снижающим точность, является ионосферная задержка радиосигнала. Именно из-за этого эффекта GPS может использоваться для исследования состояния ионосферы.

Для снижения ионосферной и тропосферной погрешностей измерений используются математические модели, комбинирование данных, сглаживание данных и режим DGPS\footnote{суть метода заключается в том, что измерения производятся двумя приемниками, один из которых неподвижен (для него известно истинное положение). Неподвижный приемник сравнивает свое истинное положение с положением, полученным с GPS, и отправляет поправочные коэффициенты второму приемнику.}.

Комбинация кодовых и фазовых измерений и использование их в алгоритмах сглаживания данных позволяют эффективно фильтровать погрешности, связанные с геометрией рабочего созвездия, шумами приемника, случайными отклонениями орбит часов и многолучевостью. 

\subsection{Геометрические положения, используемые для GPS зондирования}
Для расчета полного электронного содержания необходимо знать направление на спутник. На рисунке \ref{pic1}, можно увидеть схематическое представление геометрии системы <<Земля -- спутник>>.

\begin{figure}[!h]
\centering
\includegraphics[width = \linewidth]{pics/pic1.png}
\caption{Геометрия системы <<Земля -- спутник>>: $O$ -- центр Земли; $S$ -- спутник; $B$ -- пункт наблюдения; $P$ -- ионосферная точка; $P_I$ -- подионосферная точка; $P_S$ -- подспутниковая точка; $h_\text{max}$ -- высота максимума слоя F2 ионосферы. \cite{afraimovich}}
\label{pic1}
\end{figure}

Для вычисления координат $\alpha_S$, $\theta_S$, которые являются, соответственно, азимутом и углом места (элевация), используется метод расчета на основе геодезических координат спутника и точки наблюдения. С достаточной для практических целей точностью азимут и угол места могут быть вычислены с помощью формул \cite{kotyashkin}:

\begin{equation}
	\begin{aligned}
	&\alpha_S = \arccos{\left(\frac{\sin{\Phi_S} - \sin{\Phi}\cos{\psi_S}}{\sin{\sigma}\cos{\Phi}}\right)};\\
	&\theta_S = \arctan{\left(\frac{\cos{\Psi_S} - R_E/R_S}{\sin{\Psi_S}}\right)};\\
	&\Psi_S = \arccos{\left(\sin{\Phi}\sin{\Phi_S} + \cos{\Phi}\cos{\Phi_S}\cos{\left(\Lambda_S - \Lambda\right)}\right)},
	\end{aligned}
\end{equation}

где $R_S$ -- радиус орбиты спутника; $R_E$ -- радиус Земли; $\Phi, \Lambda$ -- геодезические широта и долгота точки наблюдения; $\Phi_S, \Lambda_S$ -- геодезические широта и долгота спутника; $\Psi_S$ -- центральный угол между точкой наблюдения и спутником.

Для вычисления координат ионосферной и подионосферной точек используются следующие выражения:

\begin{equation}
\begin{aligned}
	&\phi_P = \arcsin{\left(\sin{\phi_B}\cos{\psi_P} + \cos{\phi_B}\sin{\Psi_P}\cos{\alpha_S}\right)};\\
	&l_P = l_B + \arcsin{\left(\sin{\Psi_P}\sin{\alpha_S}\sec{\phi_P}\right)};\\
	&\Psi_P = \frac{\pi}{2} - \theta_S - arcsin{\left(\frac{R_E}{R_E + h_\text{max}}\cos{\theta_S}\right)},
\end{aligned}
\end{equation}

где $\phi_B, l_B$ -- географические координаты точки наблюдения; $\alpha_S, \theta_S$ -- азимут и угол места луча <<приемник -- спутник>>; $\Psi_P$ -- центральный угол между точкой наблюдения и ионосферной точкой; $\phi_P, l_P$ -- широта и долгота ионосферной точки 

\subsection{Принципы расчета ПЭС по данным GPS приемников}
\subsubsection{Определение ПЭС по двухчастотным фазовым измерениям псевдодальности}
При распространении сигнала вдоль луча <<приемник -- спутник>> возникает набег фазы, который определяется формулой \cite{devis}:

\begin{equation}
\varphi_\text{1,2} = \frac{2\pi f_{1,2}}{c} \int\limits_{0}^{D}n_{1,2}ds + \varphi_0,
\end{equation}

где $f_1 \text{и}  f_2$ -- рабочие частоты GPS; $\varphi_{1,2}$ -- набег фазы для частот $f_1, f_2$; $\varphi_0$ некоторая неизвестная начальная фаза; $n_{1,2}$ -- коэффициент преломления в ионосфере для сигналов $f_1, f_2$; $D$ -- расстояние между приемником и передатчиком.

При пренебрежении влиянием соударений и магнитного поля Земли, коэффициент преломления будет иметь вид \cite{devis}, \cite{ratcliff}:

\begin{equation}
\label{n_equ}
n_{1,2} \approx 1 - \frac{40.308N_e}{f_{1,2}^2},
\end{equation}

где $N_e$ -- локальная электронная концентрация.

Тогда выражение для набега фазы примет вид:

\begin{equation}
\varphi_{1,2} = \frac{2\pi f_{1,2}}{c}D - 40.308\frac{2\pi}{c f_{1,2}}\int\limits_{S_{bot}}^{S_{top}}N_e ds + \varphi_0,
\end{equation}

где $S_{bot} \text{и} S_{top}$ -- высота нижней и верхней границы ионосферы, соответственно. В этом равенстве величина $I = \int\limits_{S_{bot}}^{S_{top}}N_eds$ называется полным электронным содержанием.

Учитывая, что длина волны $\lambda = c / f$, а $L = \varphi / 2\pi$ -- число оборотов фазы, то уравнение можно записать как:

\begin{equation}
L_{1,2} \lambda_{1,2} = D - \frac{40.308}{f_{1,2}^2}I + \varphi_0.
\end{equation}

Из последнего выражения можно получить формулу для определения ПЭС:

\begin{equation}
\label{tecF}
I = \frac{1}{40.308}\frac{f_1^2 f_2^2}{f_1^2 - f_2^2} \left[ \left( L_1\lambda_1 - L_2\lambda_2 \right) + \text{const}_{1,2} + \sigma L \right],
\end{equation}

где $L_1\lambda_1 \text{и} L_2 \lambda_2$ -- приращения фазового пути радиосигнала, вызванные задержкой фазы в ионосфере; $L_1$ и $L_2$ -- фазовые измерения GPS-приемника на соответствующих частотах; $\text{const}_{1,2}$ -- неоднозначность фазовых измерений; $\sigma L$ -- ошибка измерения фазы.

Измерения фазы, получаемые с помощью GPS, имеют достаточно высокую точность, так как ошибка в определении ПЭС при 30-секундных интервалах усреднения не превышает $10^{14} \text{м}^{-2}$ (или 0.01 TECU). 

Единица измерения, принятая для описания ПЭС, является TECU (Total Electron Content Unit). Ее значение равно $10^{16} \text{м}^2$.

\subsubsection{Определение ПЭС по кодовым измерениям псевдодальности}
Сейчас будет рассмотрен метод определения ПЭС  по данным кодовых задержек. Групповой путь радиоволны определяется формулой \cite{devis}:

\begin{equation}
P_{1,2} = c \tau_{1,2} = \int \limits_{0}^{D} n_{1,2}' ds,
\end{equation}

где $P_{1,2}$ -- групповой путь для соответствующих частот; $\tau_{1,2}$ -- время распространения сигналов; 
$n_{1,2}' = n_{1,2} + f_{1,2} \frac{\partial n_{1,2}}{\partial f_{1,2}}$ -- групповой показатель преломления в ионосфере для соответствующих сигналов. Учитывая выражение (\ref{n_equ}):

\begin{equation}
n_{1,2}' \approx 1 + \frac{40.308 N_e}{f_{1,2}^2}.
\end{equation}

Используя две предыдущие формулы, можно получить формулу для определения ПЭС, аналогичную фазовым измерениям:

\begin{equation}
\label{tecP}
I = \frac{1}{40.308} \frac{f_1^2 f_2^2}{f_1^2 - f_2^2} \left[ \left( P_2 - P_1 \right) + \sigma P \right],
\end{equation} 

где $\sigma P $ -- ошибка измерения по псевдодальности по $P$-коду.

Стоит заметить, что ПЭС, вычисленный по формуле (\ref{tecP}), также содержит некоторую аддитивную константу, которая зависит от станции и спутника, которая, вероятнее всего, связана с частотно-зависимыми задержками в аппаратуре \cite{kozharin}. Кроме того, такие данные сильно зашумлены по сравнению с фазовыми измерениями. Рисунок \ref{pic2} демонстрирует различную зашумленность ПЭС. Из-за высокого уровня шума в данных, определенных по кодовым задержкам, делает практически невозможным выделение вариаций ПЭС, обусловленными неоднородностями электронной концентрации в ионосфере. Таким образом, в ионосферных исследованиях предпочитают использовать ПЭС, измеренный фазовым методом. 

\begin{figure}[!h]
\centering
\includegraphics[width = \linewidth]{pics/pic2.png}
\caption{Зашумленность ПЭС, вычисленного по данным измерений группового (кривые <<FFMJ>>, <<LEIJ>> и <<PTBB>>) и фазового (кривая <<Phase>>) запаздывания сигналов GPS \cite{kozharin}.}
\label{pic2}
\end{figure}

\subsubsection{Преобразование наклонного ПЭС  в вертикальное}
Измеренная по выше описанным формулам величина ПЭС пропорциональная расстоянию между спутником и приемником. В основном при исследовании ионосферных возмущений требуется некоторая нормировка амплитуда вариации ПЭС. С этой целью преобразуют полученные значения <<наклонного>> ПЭС в эквивалентное <<вертикальное>>, соответствующее углу места $\theta_S = 90^{\circ}$ 

Учитывая модель сферичной Земли, формула преобразования имеет вид \cite{klobuchar}:

\begin{equation}
I_V = I \cos{\left[ \arcsin{\left( \frac{R_E}{R_E + h_\text{max}} \cos{\theta_s}\right)} \right]},
\end{equation} 

где $I_V$ -- вертикальное значение ПЭС. 

\newpage
\printbibliography

\end{document}
